\RequirePackage{amsmath}
\RequirePackage{ulem}
\newcommand{\VSTUInitializeTZ} {
\renewcommand{\VSTUDocumentType}{91}
%--------------CHAPTER REDEFINTION----
\renewcommand\chapter{%
\vspace{\GostBeforTitleSkip}
  \addtocounter{totalfigure}{\c@figure}\setcounter{figure}{0}%
  \addtocounter{totaltable} {\c@table }\setcounter{table}{0}%
  \@startsection{chapter}{0}%
  {\Gost@ChapterIndent}{0mm}{\GostAfterTitleSkip}%
  {\GostTitleStyle\normalfont\Gost@ChapterStyle}%
}
%--------------APPENDIX REDEFINTION----
\gdef\theappendix{\VSTUDocumentNumbersPrefix\@arabic\c@appendix}%
\renewcommand{\appendix}[1]{
  \clearpage
  \addtocounter{appendix}{1}

  % Грязный хак нумерации рисунков\таблиц внутри приложений.
  % Сначала добавляем figure к totalfigure, а затем сбрасываем figure в 0.
  % Аналогично и для table.
  \addtocounter{totalfigure}{\c@figure}%
  \addtocounter{totaltable}{\c@table}%
  \setcounter{figure}{0}%
  \setcounter{table}{0}%

  % Внешний вид номеров рисунков: сначала номер приложения,
  % за ним номер рисунка\таблицы внутри приложения.
  \renewcommand{\thefigure}{\theappendix.\arabic{figure}}%
  \renewcommand{\thetable}{\theappendix.\arabic{table}}%

  \begin{center}%
  \appendixname~\theappendix~{##1}%
  \end{center}%
  \addcontentsline{toc}{chapter}{\appendixname~\theappendix~{##1}}%
}%

{
\sloppy
%---------------TZ_TITLE---------------
\clearpage
\thispagestyle{empty}
\begin{center}
Министерство образования и науки Российской Федерации\\
Федеральное государственное бюджетное образовательное учреждение\\
высшего профессионального образования\\
Волгоградский Государственный Технический Университет\\
Кафедра <<Программное обеспечение автоматизированных систем>>\\
\end{center}
\vfill
\hfill
\begin{minipage}[c]{18em}
Утверждаю\\
\VSTUHeadOfDepartmentPost\\
\makebox[2cm]{\hrulefill}\VSTUHeadOfDepartmentDegree~\VSTUHeadOfDepartmentName\\
<<\makebox[1.5cm]{\hrulefill}>>\makebox[3.5cm]{\hrulefill}\the\year
\end{minipage}
\vspace{8mm}
\begin{center}
<<\VSTUTitle>>\\
\vspace{\fill}
ТЕХНИЧЕСКОЕ ЗАДАНИЕ\\
\vspace{8mm}
\VSTUDocumentCode{}\\
\vspace{8mm}
Листов \totalpages\\
\vspace{15mm}
\end{center}
\begin{flushright}
\begin{minipage}[c]{15em}
Научный руководитель\\
\VSTUDirectorDegree{}\\
\makebox[2cm]{\hrulefill}\VSTUDirectorName\\
<<\makebox[1.5cm]{\hrulefill}>>\makebox[3.5cm]{\hrulefill}\the\year
\end{minipage}
\end{flushright}
\vspace{8mm}

\begin{flushleft}
\begin{minipage}[c]{15em}
Нормоконтролер\\
\VSTUStandardsAdviserDegree\\
\makebox[2cm]{\hrulefill}\VSTUStandardsAdviserName\\
<<\makebox[1.5cm]{\hrulefill}>>\makebox[3.5cm]{\hrulefill}\the\year
\end{minipage}
\hfill
\begin{minipage}[c]{15em}
Исполнитель\\
студент\ группы\ \VSTUStudentGroup\\
\makebox[2cm]{\hrulefill}\VSTUStudentName\\
<<\makebox[1.5cm]{\hrulefill}>>\makebox[3.5cm]{\hrulefill}\the\year
\end{minipage}
\end{flushleft}

\vspace{\fill}
\begin{center}
Волгоград,\ \the\year
\end{center}
}
}
