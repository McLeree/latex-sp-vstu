\RequirePackage{amsmath}
\RequirePackage{ulem}
\newcommand{\VSTUTitleHeading}{
Министерство образования и науки Российской Федерации\\
Федеральное государственное бюджетное образовательное учреждение высшего профессионального образования\\
Волгоградский государственный технический университет\\
}

% подчеркивание с подписью/датой и подчеркивание с ФИО
% #1 - длина подчеркивания
% #2 - подстрочная подпись
% #3 - надстрочная подпись
\newcommand{\VSTUUnderlinedField}[3]{%
$\underset{\text{(#2)}}{\uline{\makebox[#1]{#3}}}$%
}%

% Переопределение нумерации страниц (сверху под кодом документа)
\newcommand{\VSTUPageNumbersToTheTop}{
\fancypagestyle{plainhf} {
\fancyhf{}
\fancyhf[HCO,HCE]{\mdseries{\VSTUDocumentCode\\}\thepage\vspace{2mm}}
}
}

% Box с подписью босса
\newcommand{\VSTUTitleHeadApproval}{
\begin{minipage}[c]{18em}
УТВЕРЖДАЮ\\
\VSTUHeadOfDepartmentPost\\
\VSTUUnderlinedField{2cm}{подпись}{\vphantom{\VSTUHeadOfDepartmentDegreeAndName}}\hfill\VSTUUnderlinedField{6.5cm}{инициалы, фамилия}{\VSTUHeadOfDepartmentDegreeAndName}\\
<<\makebox[1.5cm]{\hrulefill}>>\makebox[3.5cm]{\hrulefill}\the\year
\end{minipage}
}

% Box с подписью руководителя
\newcommand{\VSTUTitleDirector}{
\begin{minipage}[c]{15em}
Научный руководитель\\
\VSTUDirectorDegreeAndPost\\
\makebox[2cm]{\hrulefill} \VSTUDirectorName\\
<<\makebox[1.5cm]{\hrulefill}>>\makebox[3.5cm]{\hrulefill}\the\year
\end{minipage}
}

% Box с подписью нормоконтролера
\newcommand{\VSTUTitleStandardsAdviser}{
\begin{minipage}[c]{15em}
Нормоконтролер\\
\VSTUStandardsAdviserDegreeAndPost\\
\makebox[2cm]{\hrulefill} \VSTUStandardsAdviserName\\
<<\makebox[1.5cm]{\hrulefill}>>\makebox[3.5cm]{\hrulefill}\the\year
\end{minipage}
}

% Box с подписью исполнителя (студента)
\newcommand{\VSTUTitleImplementer}{
\begin{minipage}[c]{15em}
Исполнитель\\
студент~группы~\VSTUStudentGroup\\
\makebox[2cm]{\hrulefill} \VSTUStudentName\\
<<\makebox[1.5cm]{\hrulefill}>>\makebox[3.5cm]{\hrulefill}\the\year
\end{minipage}
}

\newcommand{\VSTUInitializeTZ} {
\renewcommand{\VSTUDocumentCodeSuffix}{91}
\newcommand{\VSTULUCode}{А.В.00001--01~\VSTUDocumentCodeSuffix~01--1}
% Переопределение команды раздела
\renewcommand\chapter{%
  \vspace{\GostBeforeTitleSkip}
  \@startsection{chapter}{0}%
  {\Gost@ChapterIndent}{0mm}{\GostAfterTitleSkip}%
  {\GostTitleStyle\normalfont\Gost@ChapterStyle}%
}
% Переопределение команды приложений
\gdef\theappendix{\VSTUDocumentNumbersPrefix\@arabic\c@appendix}%
\renewcommand{\appendixcommon}[1]{
  \addtocounter{appendix}{1}
  \addtocounter{totalappendices}{1}

  % Грязный хак нумерации рисунков/таблиц внутри приложений.
  % Сначала добавляем figure к totalfigures, а затем сбрасываем figure в 0.
  % Аналогично и для table.
  \addtocounter{totalfigures}{\c@figure}%
  \addtocounter{totaltables}{\c@table}%
  \addtocounter{totalequations}{\c@equation}%
  \setcounter{figure}{0}%
  \setcounter{table}{0}%
  \setcounter{equation}{0}%

  % Внешний вид номеров рисунков: сначала номер приложения,
  % за ним номер рисунка/таблицы внутри приложения.
  \renewcommand{\thefigure}{\theappendix.\arabic{figure}}%
  \renewcommand{\thetable}{\theappendix.\arabic{table}}%
  \renewcommand{\theequation}{\theappendix.\arabic{equation}}%

  \addcontentsline{toc}{chapter}{\appendixname~\theappendix~---~{##1}}%
}%
{
% Титульник ТЗ раз
\clearpage
\thispagestyle{empty}
\addtocounter{page}{-1}
\begin{center}
\VSTUTitleHeading
Кафедра <<\VSTUDepartment>>\\
\end{center}
\vspace{\fill}
\hfill\VSTUTitleHeadApproval
\vspace{8mm}
\begin{center}
<<\VSTUTitle>>\\
\vspace{\fill}
ТЕХНИЧЕСКОЕ ЗАДАНИЕ\\
\vspace{8mm}
\VSTUDocumentCode\\
\vspace{8mm}
Листов \totalpages\\
\vspace{8mm}
\end{center}
\begin{flushright}
\VSTUTitleDirector
\end{flushright}
\vspace{8mm}
\begin{flushleft}
\VSTUTitleStandardsAdviser
\hfill
\VSTUTitleImplementer
\end{flushleft}
\vspace{\fill}
\begin{center}
Волгоград,~\the\year
\end{center}
\newpage
% Титульник ТЗ два
\clearpage
\thispagestyle{empty}
\hfill\VSTUTitleHeadApproval
\vspace{\fill}
\begin{center}
\VSTUTitle\\
\vspace{8mm}
ЛИСТ УТВЕРЖДЕНИЯ\\
\VSTUDocumentCode\\
\VSTULUCode--ЛУ\\
Листов 1\\
\vspace{\fill}
\end{center}
\begin{flushright}
\VSTUTitleStandardsAdviser
\vfill
\VSTUTitleImplementer
\end{flushright}
\vspace{\fill}
\begin{center}
Волгоград,~\the\year
\end{center}
\newpage
}
}
