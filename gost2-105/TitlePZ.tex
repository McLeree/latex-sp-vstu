\RequirePackage{amsmath}
\RequirePackage{ulem}
\newcommand{\VSTUInitializePZ}{
    \pagestyle{empty}
    % Титульник ПЗ раз
    % Это титульник на всю диссертацию и он не нумеруется
    \clearpage
    \begin{center}
    \VSTUTitleHeading
    \end{center}
    Факультет~\uline{\VSTUFaculty\hfill}\\
    Кафедра~\uline{\VSTUDepartment\hfill}\\
    \vfill
    \hfill\VSTUTitleHeadApproval
    \vfill
    \begin{center}
    ПОЯСНИТЕЛЬНАЯ ЗАПИСКА
    \end{center}
    к \VSTUUnderlinedField{26em}{наименование вида работы}{\VSTUDocumentTypeDative}
    \hfill
    на тему\\
    \VSTUTitleUL\\
    Автор \VSTUUnderlinedField{6cm}{подпись и дата подписания}{\vphantom{\VSTUStudentFullName}}
    \hfill
    \VSTUUnderlinedField{8cm}{фамилия, имя, отчество}{\VSTUStudentFullName}\\
    Обозначение \VSTUUnderlinedField{\width}{код документа}{\VSTUDocumentCode}\\
    Группа \VSTUUnderlinedField{3cm}{шифр группы}{\VSTUStudentGroup}\\
    Направление \VSTUUnderlinedField{25.9em}{код по ОКСО, наименование направления, программы}{\VSTUDirection}\\
    \uline{\makebox[\textwidth]{\hfill}}\\
    Руководитель работы \VSTUUnderlinedField{6cm}{подпись и дата подписания}{\vphantom{\VSTUDirectorDegreeAndName}}
    \hfill
    \VSTUUnderlinedField{5cm}{инициалы и фамилия}{\VSTUDirectorDegreeAndName}\\
    Консультанты по разделам:\\
    \VSTUUnderlinedField{5.5cm}{краткое наименование раздела}{}
    \hfill
    \VSTUUnderlinedField{5cm}{подпись и дата подписания}{}
    \hfill
    \VSTUUnderlinedField{4cm}{инициалы и фамилия}{}\\
    \VSTUUnderlinedField{5.5cm}{краткое наименование раздела}{}
    \hfill
    \VSTUUnderlinedField{5cm}{подпись и дата подписания}{}
    \hfill
    \VSTUUnderlinedField{4cm}{инициалы и фамилия}{}\\
    %\VSTUUnderlinedField{5.5cm}{краткое наименование раздела}{}
    %\hfill
    %\VSTUUnderlinedField{5cm}{подпись и дата подписания}{}
    %\hfill
    %\VSTUUnderlinedField{4cm}{инициалы и фамилия}{}\\
    \vspace{\fill}\\
    Нормоконтролер \VSTUUnderlinedField{6cm}{подпись и дата подписания}{\vphantom{\VSTUStandardsAdviserDegreeAndName}}
    \hfill
    \VSTUUnderlinedField{5cm}{инициалы и фамилия}{\VSTUStandardsAdviserDegreeAndName}\\
    \vspace{\fill}
    \begin{center}
    Волгоград,~\the\year
    \end{center}
    \clearpage
    % Титульник ПЗ два
    \newcounter{pageAtSecondTitle}
    \setcounter{pageAtSecondTitle}{\thepage}
    \begin{center}
    \VSTUTitleHeading
    \end{center}
    Кафедра~\uline{\VSTUDepartment\hfill}\\
    \vfill
    \hfill\VSTUTitleHeadApproval
    \vspace{\fill}
    \\Задание на \VSTUUnderlinedField{26.9em}{наименование вида работы}{\VSTUDocumentTypeGenitive}\\
    Студент \VSTUUnderlinedField{28em}{фамилия, имя, отчество}{\VSTUStudentFullName}\\
    Код кафедры \uline{\makebox[4cm]{\VSTUDepartmentCode}} \hfill Группа \uline{\makebox[4cm]{\VSTUStudentGroup}}\\
    \vspace{1mm}\\
    Тема \VSTUTitleUL\\
    Утверждена приказом по университету от <<\uline{\makebox[0.5cm]{\VSTUOrderDate}}>> \uline{\makebox[1.8cm]{\VSTUOrderMonth}} \uline{\makebox[1.2cm]{\VSTUOrderYear}} \textnumero\uline{\makebox[2cm]{\VSTUOrderNumber}}\\
    Срок представления готовой работы <<\uline{\makebox[0.5cm]{\VSTUDeadlineDate}}>> \uline{\makebox[2cm]{\VSTUDeadlineMonth}} \uline{\makebox[1.2cm]{\the\year}} \VSTUUnderlinedField{35mm}{подпись студента}{}\\
    Исходные данные для выполнения работы\\
    \VSTUInitialDataUL\\
    \vspace{4mm}\\
    Содержание основной части пояснительной записки
    {\small
    \VSTUPZContents
    }
    \noindent Перечень графического материала
    {\small
    \VSTUPZGraphics
    }
    \vspace{\fill}
    \noindent Руководитель работы \VSTUUnderlinedField{6cm}{подпись и дата подписания}{\vphantom{\VSTUDirectorDegreeAndName}}
    \hfill
    \VSTUUnderlinedField{5cm}{инициалы и фамилия}{\VSTUDirectorDegreeAndName}\\\\
    Консультанты по разделам:\\
    \VSTUUnderlinedField{5.5cm}{краткое наименование раздела}{}
    \hfill
    \VSTUUnderlinedField{5cm}{подпись и дата подписания}{}
    \hfill
    \VSTUUnderlinedField{4cm}{инициалы и фамилия}{}\\
    %\VSTUUnderlinedField{5.5cm}{краткое наименование раздела}{}
    %\hfill
    %\VSTUUnderlinedField{5cm}{подпись и дата подписания}{}
    %\hfill
    %\VSTUUnderlinedField{4cm}{инициалы и фамилия}{}\\
    %\VSTUUnderlinedField{5.5cm}{краткое наименование раздела}{}
    %\hfill
    %\VSTUUnderlinedField{5cm}{подпись и дата подписания}{}
    %\hfill
    %\VSTUUnderlinedField{4cm}{инициалы и фамилия}{}\\
    \clearpage
    % Титульник ПЗ три
    % Теперь добавляем суффикс к коду МД
    \renewcommand{\VSTUDocumentCodeSuffix}{81}
    % Уменьшаем счетчик страниц, поскольку были двухсторонние страницы
    \newcounter{pagesLost}
    \setcounter{pagesLost}{\thepage}
    \addtocounter{pagesLost}{-\value{pageAtSecondTitle}} % теперь в этом счетчике количество страниц (не листов) во втором титульнике
    \ifthenelse{\isodd{\value{pagesLost}}}{\addtocounter{pagesLost}{1}}{} % если количество страниц нечетное, добавляем 1 фиктивную
    \dividecounter{pagesLost}{2} % финальное значение счетчика - делим пополам
    \addtocounter{page}{-\value{pagesLost}}
    \begin{center}
    \VSTUTitleHeading
    Кафедра <<\VSTUDepartment>>\\
    \end{center}
    \vfill
    \hfill\VSTUTitleHeadApproval
    \vspace{8mm}
    \begin{center}
    \VSTUTitle\\
    \vspace{\fill}
    ПОЯСНИТЕЛЬНАЯ ЗАПИСКА\\
    \vspace{8mm}
    \VSTUDocumentCode\\
    \vspace{8mm}
    Листов \totalpages\\
    \vspace{\fill}
    \end{center}
    \begin{flushright}
    \VSTUTitleDirector
    \end{flushright}
    \vspace{8mm}
    \begin{flushleft}
    \VSTUTitleStandardsAdviser
    \hfill
    \VSTUTitleImplementer
    \end{flushleft}
    \vspace{\fill}
    \begin{center}
    Волгоград,~\the\year
    \end{center}
    \clearpage
    \pagestyle{plainhf}
}
